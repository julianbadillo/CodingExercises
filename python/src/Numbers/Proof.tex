\documentclass[12pt]{article}

\usepackage[spanish]{babel}
\usepackage[utf8x]{inputenc}
\usepackage{amsmath}
\usepackage{graphicx}

\title{Your Paper}
\author{You}

\begin{document}
\maketitle

\begin{abstract}
Your abstract.
\end{abstract}

\section{Un Ejercicio...}

Puede usted calcular la siguiente cantidad?
$$2^{2^{2006}} \bmod{p} \equiv ?$$

Tal vez a primera vista uno pensaría en calcular el exponente $2^{2006}$ 
y después $2^{2^{2006}}$ para, como última operación calcular el módulo.\\

Sin embargo si hacemos el cálculo de esta manera vamos a ver que:
$$ 2 ^{2006} = 73480364497552289550901324875371646977428332933676492830569135156...$$... Un número de 2006 bits.\\
Si tratamos de hacer 2 elevado a ese número, será un número tan grande que no
nos cabrá en la memoria de ningún computador (probablemente no quepa en la
memoria de todos los computadores del planeta combinados).\\
Cómo podríamos reducir este exponente para calcularlo más fácil? Para esto vamos
a usar el pequeño Teorema de Fermat, observe a continuación.\\
\paragraph{Pequeño Teorema de Fermat:}
	Sea $p$ un número primo. Se tiene que para cualquier $a$, entero positivo:
    $$a^{p-1} \equiv 1 \bmod{p}$$

\section{Conjetura}
Ahora, el Pequeño Teorema de Fermat no nos sirve directamente para calcular el
resultado, por lo cuál vamos a plantear una conjetura:

\paragraph{Conjetura} Sea $p$ un número primo y $a$, $b$ dos números enteros. Se
tiene que:
$$a^{b} \equiv a^{b \bmod{p-1}} \bmod{p}$$

Ahora, esta conjetura nos serviría para calcular el resultado que queremos porque
hacer una operación modular antes de hacer el exponente es más práctico.\\
Sin embargo para poder usarlo, tenemos que demostrarlo.

\section{Demostración}

Partamos de la conjetura que queremos demostrar:
$$a^{b} \equiv a^{b \bmod{p-1}} \bmod{p}$$
Sabemos que al hacer $b \bmod{p-1}$, en realidad estamos haciendo el residuo de
una división entera. Este residuo lo podemos expresar como:
$$b \mod{p-1} = b - k\cdot(p-1)$$
Donde $k$ es un entero positivo.\\
Si reemplazamos este resultado en nuestra conjetura:
$$a^{b} \equiv a^{b - k\cdot(p-1)} \bmod{p}$$
Multiplicando por $a^{k\cdot(p-1)}$ ambos lados de la ecuación:
$$a^{b} \cdot a^{k\cdot(p-1)} \equiv a^{b - k\cdot(p-1)} \cdot a^{k\cdot(p-1)} \bmod{p}$$
Ahora es cuestión de manipular algebráicamente
$$a^{b + k\cdot(p-1)} \equiv a^{b - k\cdot(p-1) + k\cdot(p-1)} \bmod{p}$$
$$a^{b + k\cdot(p-1)} \equiv a^{b} \bmod{p}$$
$$a^{b} \cdot (a^{p-1})^k \equiv a^{b} \bmod{p}$$
Y aquí viene donde usamos el teorema de Fermat, para eliminar $a^{p-1}$
$$a^{b} \cdot (1)^k \equiv a^{b} \bmod{p}$$
$$a^{b} \cdot 1 \equiv a^{b} \bmod{p}$$
Lo cual es evidentemente cierto.
\end{document}